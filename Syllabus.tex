% Don't touch this %%%%%%%%%%%%%%%%%%%%%%%%%%%%%%%%%%%%%%%%%%%
\documentclass[11pt]{article}
\usepackage{fullpage}
\usepackage[left=1in,top=1in,right=1in,bottom=1in,headheight=3ex,headsep=3ex]{geometry}
\usepackage{graphicx}
\usepackage{float}

\newcommand{\blankline}{\quad\pagebreak[2]}
%%%%%%%%%%%%%%%%%%%%%%%%%%%%%%%%%%%%%%%%%%%%%%%%%%%%%%%%%%%%%%

% Modify Course title, instructor name, semester here %%%%%%%%

\title{R\,\LaTeX{}101T: Five Days Workshop on Effective Tools for Academic \& Research Documentation}
\author{Department of Mathematics, SSN College of Engineering}
\date{February-March, 2022}

%%%%%%%%%%%%%%%%%%%%%%%%%%%%%%%%%%%%%%%%%%%%%%%%%%%%%%%%%%%%%%

% Don't touch this %%%%%%%%%%%%%%%%%%%%%%%%%%%%%%%%%%%%%%%%%%%
\usepackage[sc]{mathpazo}
\linespread{1.05} % Palatino needs more leading (space between lines)
\usepackage[T1]{fontenc}
\usepackage[mmddyyyy]{datetime}% http://ctan.org/pkg/datetime
\usepackage{advdate}% http://ctan.org/pkg/advdate
\newdateformat{syldate}{\twodigit{\THEMONTH}/\twodigit{\THEDAY}}
\newsavebox{\MONDAY}\savebox{\MONDAY}{Mon}% Mon
\newcommand{\week}[1]{%
%  \cleardate{mydate}% Clear date
% \newdate{mydate}{\the\day}{\the\month}{\the\year}% Store date
  \paragraph*{\kern-2ex\quad #1, \syldate{\today} - \AdvanceDate[4]\syldate{\today}:}% Set heading  \quad #1
%  \setbox1=\hbox{\shortdayofweekname{\getdateday{mydate}}{\getdatemonth{mydate}}{\getdateyear{mydate}}}%
  \ifdim\wd1=\wd\MONDAY
    \AdvanceDate[7]
  \else
    \AdvanceDate[7]
  \fi%
}
\usepackage{setspace}
\usepackage{multicol}
%\usepackage{indentfirst}
\usepackage{fancyhdr,lastpage}
\usepackage{url}
\pagestyle{fancy}
\usepackage{hyperref}
\usepackage{lastpage}
\usepackage{amsmath}
\usepackage{layout}

\lhead{}
\chead{}
%%%%%%%%%%%%%%%%%%%%%%%%%%%%%%%%%%%%%%%%%%%%%%%%%%%%%%%%%%%%%%

% Modify header here %%%%%%%%%%%%%%%%%%%%%%%%%%%%%%%%%%%%%%%%%
\rhead{\footnotesize Workshop on Effective Tools for Academic Documentation}

%%%%%%%%%%%%%%%%%%%%%%%%%%%%%%%%%%%%%%%%%%%%%%%%%%%%%%%%%%%%%%
% Don't touch this %%%%%%%%%%%%%%%%%%%%%%%%%%%%%%%%%%%%%%%%%%%
\lfoot{}
\cfoot{\small \thepage/\pageref*{LastPage}}
\rfoot{}

\usepackage{array, xcolor}
\usepackage{color,hyperref}
\definecolor{clemsonorange}{HTML}{EA6A20}
\hypersetup{colorlinks,breaklinks,linkcolor=clemsonorange,urlcolor=clemsonorange,anchorcolor=clemsonorange,citecolor=black}

\begin{document}

\maketitle

\blankline

\begin{tabular*}{.93\textwidth}{@{\extracolsep{\fill}}lr}

%%%%%%%%%%%%%%%%%%%%%%%%%%%%%%%%%%%%%%%%%%%%%%%%%%%%%%%%%%%%%%

% Modify information %%%%%%%%%%%%%%%%%%%%%%%%%%%%%%%%%%%%%%%%%
E-mail: \texttt{Sundarakannanm@ssn.edu.in} & Web: \href{https://www.ssn.edu.in/college-of-engineering/mathematics-department-ssn-institutions/}{\tt\bf www.ssn.edu.in}  \\

 Office Hours: M-F 12.30-1:30 pm  &  Class Hours: M-F 9.00 am-4:15pm \\

 Office: AB 303/AB305 & Class Room: RB 202/ CLC 303 \\
 & \\
Lab Room: SAILAB & Lab Hours: W 9.30 am-4.30 pm \\
&\\
Contact: +91 9944558794 &  \\
\hline
\end{tabular*}

\vspace{5 mm}

% First Section %%%%%%%%%%%%%%%%%%%%%%%%%%%%%%%%%%%%%%%%%%%%

\section*{Course Description}

{\em Workshop on Effective Tools for Academic Documentation} is a five days hands-on training programme specially designed for academicians and researchers. In this course advance statistical tool- R and the popular scientific document preparation tool -\LaTeX{} will be discussed. 

\bigskip

\noindent This course will be executed in the form of an online workshop. There will be short quiz, assignments and final project reports of a selected themes. The participants with at least 75\% score in the assessments will be eligible for an e-certificate.

% Second Section %%%%%%%%%%%%%%%%%%%%%%%%%%%%%%%%%%%%%%%%%%%

\section*{Required Materials}

\begin{itemize}
\item Course notes available on Canvas LMS. A laptop/ desktop with internet connectivity is the main requirement for this workshop.
\end{itemize}

% Third Section %%%%%%%%%%%%%%%%%%%%%%%%%%%%%%%%%%%%%%%%%%%

\section*{Prerequisites/Co-requisites}
Prerequisites: Basic computer knowledge, fundamentals of descriptive statistics, basic knowledge in typesetting of academic documents.
% Fourth Section %%%%%%%%%%%%%%%%%%%%%%%%%%%%%%%%%%%%%%%%%%%

\section*{Course Objectives}
Upon successful completion of this workshop the participant will be able to:
\begin{enumerate}
\item Learn R programming for EDA
\item Use R software for statistical analysis of research data
\item Learn the advanced typesetting architecture for scientific documents
\item Use \LaTeX{} to prepare journal articles, academic reports and thesis
\item Practice academic process automation
\end{enumerate}

% Fifth Section %%%%%%%%%%%%%%%%%%%%%%%%%%%%%%%%%%%%%%%%%%%

\section*{Course Structure}
{\bf Syllabus}\\

\noindent{\bf Module-1}\\
Introduction to R programming- structure of R, basic functions, mathematical operations\\

\noindent{\bf Module-2}\\
Data processing, exploratory data analysis (EDA)- tools for visualization.\\

\noindent{\bf Module-3}\\
R for academic documentation- concept of note book and markdowns, creating scientific note books using R, R markdown options, Presentations- beamer, ioslides, html5 and  slidy presentations. Report generation- html, pdf and word formats.\\

\noindent{\bf Module-4}\\
Academic documentation- frame work for academic documents, advanced tools for academic typesetting- \LaTeX{}, BibTeX, jabRef. Basics of \LaTeX{} typesetting. Mathematical drawings with \LaTeX{}.\\

\noindent {\bf Module-5}\\
  Merging \LaTeX{} with R- Generating high quality academic documents- article, report, book, leaflets, presentations. Academic Process Automation using R and \LaTeX{}.
\subsection*{Class Structure}


\bigskip

\begin{description}
\item[Day 1:] Module-1 in hands-on training format. There will be four sessions- two in forenoon and two in afternoon.
\item[Day 2:] Module-2 in hands-on training format. There will be four sessions- two in forenoon and two in afternoon.
\item[Day 3:] Module-3 in hands-on training format. There will be four sessions- two in forenoon and two in afternoon.
\item[Day-4]Module-4 in hands-on training format. There will be four sessions- two in forenoon and two in afternoon.
\item[Day-5] Module-5 in hands-on training format. There will be four sessions- two in forenoon and two in afternoon.
\end{description}

\subsection*{Assessments}

Assignments and quiz will be available through the course website. Participants are expected to complete them after the regular class sessions.

\subsubsection*{Hands-on sessions}
After each session, the participants are expected to submit their work in the form of `html`/ `pdf' file in the Canvas LMS.

There will be short quiz after each session through Canvas.
\subsubsection*{Lab}
The lab sessions will be in R / \LaTeX{} cloud platforms. The markdown files should be submitted through Canvas for evaluation.

\subsubsection*{Class Project}
Prepare a report using R/ \LaTeX{} based on the theme given by the course instructor. The soft copy of the report should be submitted through Canvas for evaluation.

\subsection*{Grading Policy}
The typical Canvas grading scale will be used. The organizers reserve the right to curve the scale dependent on overall class scores at the end of the course. Any curve will only ever make it easier to obtain a certain letter grade. The grade will count the assessments using the following proportions:
\begin{itemize}
	\item \underline{\textbf{60\%}} of your grade will be determined by submission of course work sheets (20\% each).
	\item \underline{\textbf{20\%}} of your grade will be determined by performance in quiz.
	\item \underline{\textbf{20\%}} of your grade will be determined by your end project report submission.
\end{itemize}

% Add a figure %%%%%%%%%%%%%%%%%%%%%%%%%%%%%%%%%%%%%%%%%%%

%\begin{figure*}
%\includegraphics[width=1.3\textwidth,angle=90]{Concept_map_315.pdf}
%\end{figure*}

% Fifth Section %%%%%%%%%%%%%%%%%%%%%%%%%%%%%%%%%%%%%%%%%%%

%\newpage
\section*{Course Policies}

%\subsection*{During Class}
%\footnotesize{We understand that the electronic recording of notes will be important for class and so computers will be allowed in class. Please refrain from using computers for anything but activities related to the class. Phones are prohibited as they are rarely useful for anything in the course.}

\subsection*{Attendance Policy}
\footnotesize{Attendance is expected in all hands-on and lab sections.}% Valid excuses for absence will be accepted before class. In extenuating circumstances, valid excuses with proof will be accepted after class}

\subsection*{Policies on Incomplete Grades and Late Assignments}
\footnotesize{If an extended deadline is not authorized by the instructor or the course coordinator, an unfinished incomplete grade will automatically change to an F after proper notifications. `Incompletes' that change to F will count as an attempted assignment on transcripts. The burden of fulfilling an incomplete grade is the responsibility of the participant.}

\footnotesize{Late assignments will be accepted for no penalty if a valid excuse is communicated to the instructor before the deadline. After the deadline, assignments will be accepted for a 50\% deduction to the score up to 1 day after the deadline. After this any assignments handed in will be given 0.}

\subsection*{Academic Integrity and Honesty}
\footnotesize{Participants are required to comply with the institution policy on academic integrity. Don't cheat.}




% Course Schedule %%%%%%%%%%%%%%%%%%%%%%%%%%%%%%%%%%%%%%%%%%%

%\newpage
%\section*{Schedule and weekly learning goals}


\end{document}

