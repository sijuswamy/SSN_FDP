\documentclass[a4paper,10pt]{report}
\usepackage{graphicx}
\usepackage{index}%package for creating index page
\makeindex
\usepackage{lastpage}
\usepackage{kpfonts}
\usepackage[left=2.5cm,right=2.5cm,top=2.5cm,bottom=2.5cm]{geometry}
\usepackage{amsmath,amssymb,amsthm}
%setting chapter title style
%Options: Sonny, Lenny, Glenn, Conny, Rejne, Bjarne, Bjornstrup
\usepackage[Glenn]{fncychap}
%setting page header and footer
\usepackage{fancyhdr}
\pagestyle{fancy}
\fancyhf{}
\rhead{SSN College of Engineering}
\lhead{FDP}
\rfoot{Page \thepage~ of~ \pageref{LastPage}}
% to avoid hyphenation
\tolerance=1
	\emergencystretch=\maxdimen
	\hyphenpenalty=10000
	\hbadness=10000
	\sloppy
\title{Working with Bibliography}
\author{ss}
\renewcommand\bibname{Reference}%comment this line to get bibliography title "Bibliography"

\begin{document}
\maketitle
\chapter{Introduction}
A detailed view of reference \index{Machine learning} on R is available at \cite{bonaccorso2017machine}
\newpage
\section{A sample Section}
A sample section can be created with the command \verb+\section{section name}+\index{Section naming}.
\subsection{A Sample Subsection}

% \nocite{*}% uncomment to show all references
\bibliographystyle{IEEEtran}
\bibliography{reference}
\addcontentsline{toc}{chapter}{References}
\newpage
\printindex
\end{document}