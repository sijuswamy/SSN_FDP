\documentclass[twoside]{article}
\usepackage{amsmath}
\usepackage{amssymb}
\usepackage{multicol}
%\usepackage{lipsum} % Package to generate dummy text throughout this template
\usepackage{graphicx}
\usepackage[sc]{mathpazo} % Use the Palatino font
\usepackage[T1]{fontenc} % Use 8-bit encoding that has 256 glyphs
\linespread{1.05} % Line spacing - Palatino needs more space between lines
\usepackage{microtype} % Slightly tweak font spacing for aesthetics
\usepackage[hmarginratio=1:1,top=20mm,columnsep=20pt,left=19mm]{geometry} % Document margins
%\usepackage[left=1.5cm,right=1.5cm,top=1.5cm,bottom=1.5cm]{geometry}
\usepackage{multicol} % Used for the two-column layout of the document
\usepackage[hang, small,labelfont=bf,up,textfont=it,up]{caption} % Custom captions under/above floats in tables or figures
\usepackage{booktabs} % Horizontal rules in tables
\usepackage{float} % Required for tables and figures in the multi-column environment - they need to be placed in specific locations with the [H] (e.g. \begin{table}[H])
\usepackage{hyperref} % For hyperlinks in the PDF

\usepackage{lettrine} % The lettrine is the first enlarged letter at the beginning of the text
\usepackage{paralist} % Used for the compactitem environment which makes bullet points with less space between them
\usepackage{float}
\usepackage{abstract} % Allows abstract customization
\renewcommand{\abstractnamefont}{\normalfont\bfseries} % Set the "Abstract" text to bold
\renewcommand{\abstracttextfont}{\normalfont\small\itshape} % Set the abstract itself to small italic text

\usepackage{titlesec} % Allows customization of titles
\renewcommand\thesection{\Roman{section}}
\titleformat{\section}[block]{\large\scshape\centering}{\thesection.}{1em}{} % Change the look of the section titles

\usepackage{fancyhdr} % Headers and footers
\pagestyle{fancy} % All pages have headers and footers
\fancyhead{} % Blank out the default header
\fancyfoot{} % Blank out the default footer
\fancyhead[l]{\centering Assignemnt-1-- $\bullet$ 20MA 101 Linear Algebra \& Calculus $\bullet$ Module-3, Sh.No. 6} % Custom header text
\fancyfoot[RO,LE]{\thepage} % Custom footer text

%----------------------------------------------------------------------------------------
%	TITLE SECTION
%----------------------------------------------------------------------------------------

\title{\vspace{-15mm}\fontsize{20pt}{15pt}\selectfont\textbf{Applications of Triple Integrals}} % Article title
\author{
	\large
	\textsc{Faculty: Siju K.S.,Lekha Susan Jacob}\\
	\normalsize {Date:04.02.2022--3$^{th}$hour.} \\ 
	%\normalsize \href{mailto:siju.swamy@saintgits.org}{sijuswamy@gmail.com} % Your email address
	\vspace{-5mm}
}
\date{}

%----------------------------------------------------------------------------------------

\begin{document}
	
	\maketitle % Insert title
	
	\thispagestyle{fancy} % All pages have headers and footers
	\section{Workout the Following Problems}
	\begin{enumerate}
		\item Evaluate $\displaystyle \int\limits_0^1\int\limits_{-1}^{y^2}\int\limits_{-1}^zyzdxdzdy$
		\item Evaluate $\displaystyle \int_0^1\int_{y^2}^1\int_0^{1-x}xdzdxdy$
		\item Use triple integral find the volume of the solid in the first octant bounded by the coordinate planes and the  plane $3x+6y+4z=12$.
		\item Evaluate $\displaystyle \iint\limits_G e^{2x+y-z}dV$ where $0\leq x\leq 1;\quad 0\leq y\leq \ln(3);\quad 0\leq z\leq \ln(2)$
	\end{enumerate}
	
	%	REFERENCE LIST
	%----------------------------------------------------------------------------------------
	\begin{thebibliography}{99} % Bibliography - this is intentionally simple in this template
		%\bibitem{one}Erwin Kreyszig,Advanced Engineering Mathematics,8Ed. Wiley, 2004.
		%\bibitem{two}B.S.Grewal, Higher Engineering Mathematics,40Ed, Khanna Publications, 2010.
		\bibitem{three} N.P.Bali, A.J.Geoge, A Textbook of Engineering Mathematics,3Ed, University Science Press, 2012
	\end{thebibliography}
	%----------------------------------------------------------------------------------------
	
	%\end{multicols}
	
\end{document}
