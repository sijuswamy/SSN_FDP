\documentclass[10pt,a4paper]{article}
\usepackage[latin1]{inputenc}
\usepackage{amsmath}
\usepackage{amsfonts}
\usepackage{amssymb}
\usepackage{makeidx}
\usepackage{qrcode}
\usepackage{graphicx}
\usepackage[left=2.00cm, right=2.00cm, top=2.00cm, bottom=2.00cm]{geometry}
\makeindex
\author{Anonymous}
\title{Essentials of~\LaTeX~Typesetting }
\date{}
%document preamble
\begin{document}
	\maketitle
%	\tableofcontents
%	\listoffigures
%	\listoftables
%	\newpage
%\section{A Brief History}
%	\section{Introduction}
%	Common tasks in writing an article are
%\begin{enumerate}
%		\item Type setting running text
%		\item Creating enumerated list
%		\item Creating bulleted list
%		\item Inserting figures
%		\item Inserting tables
%		\item Type setting equations
%		\item Setting bibliography
%		\item Cross-referencing \& citation
%	\end{enumerate}
%\subsection{Type setting running text}\index{text formatting}
%We can typeset the contents using a keyboard connected to the computer. A \TeX~editor like {\ttfamily{TeXstudio}} or cloud based free editors like {\ttfamily{Overleaf}} will be used for creating \LaTeX source code. \LaTeX~ support various languages. As we expected we can copy contents from various sources and paste it using the popular {\ttfamily{Ctrl+c}} and {\ttfamily{Ctrl+v}} commands.\\
%
%A running text can be aligned using {\em center}, {\em flushright} and {\em flushleft} environments. Following examples will illustrate this environments.\index{text environments}
%\begin{verbatim}
%\begin{center}
%{\LARGE\bfseries{SAINTGITS College of Engineering, Kottayam}}\\
%\small{An Autonomous Engineering College Under APJ Abdul Kalam Technological University\\
% First Series B.Tech Degree Examinations, October 2020\\
% {\bfseries Course: 20MAT101 Linear Algebra \& Calculus}\\
% (Common to ALL branches)\\
%\end{center}
%Duration: Three hours\hfill Maximum marks: 100
%\end{verbatim}
%
%The output of this code is:\\
%\hrule
%\begin{center}
%	{\LARGE\bfseries{SAINTGITS College of Engineering, Kottayam}}\\
%	\small{An Autonomous Engineering College Under APJ Abdul Kalam Technological University}\\
%	First Series B.Tech Degree Examinations, October 2020\\
%	{\bfseries Course: 20MAT101 Linear Algebra \& Calculus}\\
%	(Common to ALL branches)
%\end{center}
%Duration: Three hours\hfill Maximum marks: 100\\
%\hrule
%\begin{verbatim}
%\begin{flushright}
%\Large{SAINTGITS College of Engineering, Kottayam}\\
%\end{flushright}
%\end{verbatim}
%Output of this code is:
%
%\begin{flushright}
%	\Large{SAINTGITS College of Engineering, Kottayam}
%\end{flushright}
%\begin{verbatim}
%\begin{flushleft}
%\Large\sc{SAINTGITS College of Engineering, Kottayam}
%\end{flushleft}
%\end{verbatim}
%Output of this code is:
%
%\begin{flushleft}
%	\Large\sc{SAINTGITS College of Engineering, Kottayam}
%\end{flushleft}
%\subsection{Creating enumerated list}
%A sequentially incremented numbered list is called an enumeration. The environment for creating an enumerated list is:
%\begin{verbatim}
%\begin{enumerate}
%\item  first item
%\item second item
%\item third item
%\end{enumerate}
%\end{verbatim}
%Numbering system for enumeration is set as per international standards.
%\subsection{Creating bulleted list}
%A solid dot instead number is used in the case of bulleted list. The general environment for bulleted list is:
%\begin{verbatim}
%\begin{itemize}
%\item first item
%\item second item
%\item third item
%\end{itemize}
%\end{verbatim}
%Output of this code is given bellow:
%\begin{itemize}
%	\item first item
%	\item second item
%	\item third item
%\end{itemize}
%\subsection*{Creating QR codes}
%A QR code can be created using \verb|\qrcode{content}| command. To use this command we have to include {\ttfamily qrcode} package in the document preamble. An example of QR code is:
%\begin{verbatim}
%\qrcode{http://www.saintgits.org}
%\end{verbatim}
%This command will create the following QR code.
%
%\qrcode{http://www.saintgits.org}
%\subsection{Inserting figures}
%	The following environment insert a figure in \LaTeX .
%	\begin{verbatim*}
%\begin{figure}[h]
%\centering
%\includegraphics[width=0.4\linewidth]{qrcodewebinar}
%\caption{The QR code for Webinar 1}
%\label{fig:qrcodewebinar1}
%\end{figure}
%\end{verbatim*}
%Output of this code is given bellow:
%	\begin{figure}[h]
%		\centering
%		\includegraphics[scale=0.9]{qrcodewebinar}
%		\caption{The QR code for Webinar 1}
%		\label{fig:qrcodewebinar1}
%	\end{figure}
%	
%	This figure can be referred using \verb|\ref{fig:qrcodewebinar1}| command. This will create an automatic reference number as shown in Figure \ref{fig:qrcodewebinar1}.
%	\subsection{Inserting tables}
%	A table can be inserted using {\ttfamily table} environment. General syntax of a table is:
%	\begin{verbatim}
%	\begin{table}[h]
%	\centering
%	\caption{A sample Table}
%	\label{tab:lebel_to_table}
%	\begin{tabular}{|c|c|c|c|}\hline
%	Sl.No&Column 1&Column 2& Column 3}\\\hline
%1.& c(1,1)&c(1,2)&c(1,3)\\\hline
%2.&c(2,1)&c(2,2)&c(2,3)\\\hline
%	\end{tabular}
%	\end{table}
%	\end{verbatim}
%	
%Output of this code is given bellow:
%\begin{table}[!htp]
%	\centering
%	\caption{A sample Table}
%	\label{tab:lebel_to_table}
%	\begin{tabular}{|c|c|c|c|}\hline
%		Sl.No&Column 1&Column 2& Column 3\\\hline
%	1.& c(1,1)&c(1,2)&c(1,3)\\\hline
%	2.&c(2,1)&c(2,2)&c(2,3)\\\hline
%\end{tabular}
%\end{table}
%This table can be referred using \verb|\ref{tab:lebel_to_table}| command. This reference will be shown as Table \ref{tab:lebel_to_table}.
%\subsection{Typesetting equations}
%Most important reason for using \LaTeX typesetting by the technical community is its wonderful power of generating equations and mathematical construct with high quality typeface. A running mathematical expression will be included  within dollar symbols (\$\$). For example, the source code \verb|$(x+y)^2=x^2+2xy+y^2$| will create the running equation- $(x+y)^2=x^2+2xy+y^2$. A centralized equation in a new line can be generated using double dollar delimiting as \verb|$$(x+y)^2=x^2+2xy+y^2$$|. Its output is $$(x+y)^2=x^2+2xy+y^2$$
%
%Numbered equations can be generated using the {\ttfamily equation} environment. The general syntax is:
%\begin{verbatim}
%\begin{equation}
%(x+y)^2=x^2+2xy+y^2\label{eqn:1}
%\end{equation}
%\end{verbatim}
%Output of this command will generate the following equation
%\begin{equation}
%(x+y)^2=x^2+2xy+y^2\label{eqn:1}
%\end{equation}
%
%Using the \verb|\eqref{eqn:1}| command we can refer this equation. This command will create the reference as \eqref{eqn:1}.
%Mathematical symbols are generated using \LaTeX~ macros. This macros are the English name of these symbols precedes by a backslash. For example the macro \verb|$\sin \theta$| generate the output $\sin\theta$.
%
%Aligned equations can be typeset using {\ttfamily align} environment. The general form of an aligned equation is:
%\begin{verbatim}
%\begin{align}
%(\sin\theta+\cos\theta)^2&=\sin^2\theta+2\sin\theta\cos\theta+\cos^2\theta\\
%&=1+2\sin\theta\cos\theta
%\end{align}
%\end{verbatim}
%Output of this source code is
%
%\begin{align}
%(\sin\theta+\cos\theta)^2&=\sin^2\theta+2\sin\theta\cos\theta+\cos^2\theta\nonumber\\
%&=1+2\sin\theta\cos\theta
%\end{align}
%As a special case a multi valued function can be type set using {\ttfamily case} environment. An example code is given bellow.
%\begin{verbatim}
%$$f(x)=\begin{cases}
%+x&;\quad x\geq 0\\
%-x&;\quad x<0
%\end{cases}$$
%\end{verbatim}
%
%Output of this code is:
%
%$$f(x)=\begin{cases}
%+x&;\quad x\geq 0\\
%-x&;\quad x<0
%\end{cases}$$
%\subsection{Typesetting bibliography}
%In \LaTeX~, bibliography can be typeset using the {\ttfamily thebibliography} environment. A sample is given bellow.
%\begin{verbatim}
%\begin{thebibliography}{99}
%\bibitem{citekey1} reference 1
%\bibitem{citekey2} reference 2
%\bibitem{citekey3} reference 3
%\end{thebibliography}
%\end{verbatim}
%Output of this code will generate the following output
%
%\begin{thebibliography}{99}
%	\bibitem{book:lamort} Lamport, L. (1994). LATEX: a document preparation system: user's guide and reference manual. Addison-wesley.
%	\bibitem{citekey2} reference 2
%	\bibitem{citekey3} reference 3
%\end{thebibliography}
%
%Usually the bibliography/ reference will appeared at the end of the document. A bibliography item (reference) can be cited using \verb|\cite{citekey}| command. As an example, reference 3 with citekey {\ttfamily citekey3} can be cited as \verb|citekey3|. This citation will be appeared in a square bracket as \cite{book:lamort}.
%\printindex
\end{document}